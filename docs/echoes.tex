% Options for packages loaded elsewhere
\PassOptionsToPackage{unicode}{hyperref}
\PassOptionsToPackage{hyphens}{url}
%
\documentclass[
]{book}
\usepackage{amsmath,amssymb}
\usepackage{lmodern}
\usepackage{iftex}
\ifPDFTeX
  \usepackage[T1]{fontenc}
  \usepackage[utf8]{inputenc}
  \usepackage{textcomp} % provide euro and other symbols
\else % if luatex or xetex
  \usepackage{unicode-math}
  \defaultfontfeatures{Scale=MatchLowercase}
  \defaultfontfeatures[\rmfamily]{Ligatures=TeX,Scale=1}
\fi
% Use upquote if available, for straight quotes in verbatim environments
\IfFileExists{upquote.sty}{\usepackage{upquote}}{}
\IfFileExists{microtype.sty}{% use microtype if available
  \usepackage[]{microtype}
  \UseMicrotypeSet[protrusion]{basicmath} % disable protrusion for tt fonts
}{}
\makeatletter
\@ifundefined{KOMAClassName}{% if non-KOMA class
  \IfFileExists{parskip.sty}{%
    \usepackage{parskip}
  }{% else
    \setlength{\parindent}{0pt}
    \setlength{\parskip}{6pt plus 2pt minus 1pt}}
}{% if KOMA class
  \KOMAoptions{parskip=half}}
\makeatother
\usepackage{xcolor}
\usepackage{longtable,booktabs,array}
\usepackage{calc} % for calculating minipage widths
% Correct order of tables after \paragraph or \subparagraph
\usepackage{etoolbox}
\makeatletter
\patchcmd\longtable{\par}{\if@noskipsec\mbox{}\fi\par}{}{}
\makeatother
% Allow footnotes in longtable head/foot
\IfFileExists{footnotehyper.sty}{\usepackage{footnotehyper}}{\usepackage{footnote}}
\makesavenoteenv{longtable}
\usepackage{graphicx}
\makeatletter
\def\maxwidth{\ifdim\Gin@nat@width>\linewidth\linewidth\else\Gin@nat@width\fi}
\def\maxheight{\ifdim\Gin@nat@height>\textheight\textheight\else\Gin@nat@height\fi}
\makeatother
% Scale images if necessary, so that they will not overflow the page
% margins by default, and it is still possible to overwrite the defaults
% using explicit options in \includegraphics[width, height, ...]{}
\setkeys{Gin}{width=\maxwidth,height=\maxheight,keepaspectratio}
% Set default figure placement to htbp
\makeatletter
\def\fps@figure{htbp}
\makeatother
\setlength{\emergencystretch}{3em} % prevent overfull lines
\providecommand{\tightlist}{%
  \setlength{\itemsep}{0pt}\setlength{\parskip}{0pt}}
\setcounter{secnumdepth}{5}
\usepackage{booktabs}
\usepackage{amsthm}
\makeatletter
\def\thm@space@setup{%
  \thm@preskip=8pt plus 2pt minus 4pt
  \thm@postskip=\thm@preskip
}
\makeatother
\ifLuaTeX
  \usepackage{selnolig}  % disable illegal ligatures
\fi
\usepackage[]{natbib}
\bibliographystyle{apalike}
\IfFileExists{bookmark.sty}{\usepackage{bookmark}}{\usepackage{hyperref}}
\IfFileExists{xurl.sty}{\usepackage{xurl}}{} % add URL line breaks if available
\urlstyle{same} % disable monospaced font for URLs
\hypersetup{
  pdftitle={Experiences, Coping, and Health Outcomes Study (ECHOES)},
  pdfauthor={By: Francesca (Fran) R. Querdasi (FQ), Genesis D. Flores (GF), \& Bridget L. Callaghan (BC)},
  hidelinks,
  pdfcreator={LaTeX via pandoc}}

\title{Experiences, Coping, and Health Outcomes Study (ECHOES)}
\author{By: Francesca (Fran) R. Querdasi \textbf{(FQ)}, Genesis D. Flores \textbf{(GF)}, \& Bridget L. Callaghan \textbf{(BC)}}
\date{Last Updated on: 2023-09-21}

\begin{document}
\maketitle

{
\setcounter{tocdepth}{1}
\tableofcontents
}
\hypertarget{introduction}{%
\chapter{Introduction}\label{introduction}}

This wiki contains information for the Experiences, Coping, and Health Outcomes Study (ECHOES)

\hypertarget{overview}{%
\chapter{Overview}\label{overview}}

\hypertarget{summary}{%
\section{Summary}\label{summary}}

ECHOES is a fully online study focusing on young adults' (ages 18-25) adversity experiences (early life and recent), current physical and mental health, and psychosocial risk/protective factors (e.g., coping style, alexithymia, social support).

\hypertarget{aims}{%
\section{Aims}\label{aims}}

The main project has the following aims:

\textbf{Aim 1}: Identify how adversity characteristics (type, timing) are related to psychosocial risk/protective factors and health in young adulthood.

\begin{itemize}
\item
  \textbf{Aim 1.1}: Test a 3-dimensional model of adversity based on recent theory (thread, deprivation, unpredictabiltiy), and, if this theorized model does not fit the data well, identify the best-fitting adversity dimensional model in this sample.
\item
  \textbf{Aim 1.2}: Test how adversity dimensions are related to alexithymia.
\end{itemize}

\textbf{Aim 2}: Characterize profiles/groups of young adults according to their experieces, psychosocial risk/protective factors, and health.

\hypertarget{background}{%
\section{Background}\label{background}}

WILL ADD MORE HERE

\hypertarget{study-details}{%
\chapter{Study Details}\label{study-details}}

\hypertarget{study-design}{%
\section{Study Design}\label{study-design}}

ECHOES is a fully online study that recruited entirely from Prolific. ECHOES was designed to get a relatively large sample size (target N = 400) with a brief protocol (\textasciitilde30 minutes). It is a cross-sectional study with 1 timepoint. Trhough Prolific, an equal number of male and female participants were recruited, and a ratio of White/non-White participants consistent with the United States population of young adults (212/400 or 53\% White, 188/400 or 47\% non-White).

\hypertarget{questionnaires}{%
\section{Questionnaires}\label{questionnaires}}

Questionnaires are listed below in the order they were administered to participants:

\begin{itemize}
\tightlist
\item
  \textbf{Demographics}

  \begin{itemize}
  \tightlist
  \item
    BAB Lab
  \end{itemize}
\item
  \textbf{Visceral Sensitivity Index (vsi)}

  \begin{itemize}
  \tightlist
  \item
    Adapted by FQ and BC from: Labus JS, Bolus R, Chang L, Wiklund I, Naesdal J, Mayer EA, Naliboff BD. The Visceral Sensitivity Index: development and validation of a gastrointestinal symptom-specific anxiety scale. Aliment Pharmacol Ther. 2004 Jul 1;20(1):89-97. doi: 10.1111/j.1365-2036.2004.02007.x. PMID: 15225175.
  \end{itemize}
\item
  \textbf{Rome IV IBS criteria (rome)}

  \begin{itemize}
  \tightlist
  \item
    BAB Lab
  \end{itemize}
\item
  \textbf{Patient Health Questionnaire, Depression Module (PHQ-9)}

  \begin{itemize}
  \tightlist
  \item
    Kroenke, K., Spitzer, R.L. \& Williams, J.B.W. The PHQ-9. J GEN INTERN MED 16, 606--613 (2001). \url{https://doi.org/10.1046/j.1525-1497.2001.016009606.x}
  \end{itemize}
\item
  \textbf{Generalized Anxiety Disorder 7 (GAD-7)}

  \begin{itemize}
  \tightlist
  \item
    Spitzer RL, Kroenke K, Williams JBW, Löwe B. A Brief Measure for Assessing Generalized Anxiety Disorder: The GAD-7. Arch Intern Med. 2006;166(10):1092--1097. \url{doi:10.1001/archinte.166.10.1092}
  \end{itemize}
\item
  \textbf{Somatic Symptoms (PHQ-15)}

  \begin{itemize}
  \tightlist
  \item
    Korber S, Frieser D, Steinbrecher N, \& Hiller, W. Classification characteristics of the Patient Health Questionnaire-15 for screening somatoform disorders in a primary care setting. Journal of Psychosomatic Research. 2011;71(3). \url{https://doi.org/10.1016/j.jpsychores.2011.01.006}
  \end{itemize}
\item
  \textbf{Chronic Pain}

  \begin{itemize}
  \tightlist
  \item
    Created by BAB Lab based on the following 2 sources:
  \item
    Deyo RA, Dworkin SF, Amtmann D, Andersson G, Borenstein D, Carragee E, Carrino J, Chou R, Cook K, DeLitto A, Goertz C, Khalsa P, Loeser J, Mackey S, Panagis J, Rainville J, Tosteson T, Turk D, Von Korff M, Weiner DK. Report of the NIH Task Force on research standards for chronic low back pain. J Pain. 2014 Jun;15(6):569-85. doi: 10.1016/j.jpain.2014.03.005. Epub 2014 Apr 29. PMID: 24787228; PMCID: PMC4128347.
  \item
    Dahlhamer J, Lucas J, Zelaya C, Nahin R, Mackey S, DeBar L, Kerns R, Von Korff M, Porter L, Helmick C. Prevalence of Chronic Pain and High-Impact Chronic Pain Among Adults - United States, 2016. MMWR Morb Mortal Wkly Rep.~2018 Sep 14;67(36):1001-1006. doi: 10.15585/mmwr.mm6736a2. PMID: 30212442; PMCID: PMC6146950.
  \end{itemize}
\item
  \textbf{Brief Resilient Coping Scale (BRCS)}

  \begin{itemize}
  \tightlist
  \item
    Sinclair VG, Wallston KA. The development and psychometric evaluation of the Brief Resilient Coping Scale. Assessment. 2004 Mar;11(1):94-101. doi: 10.1177/1073191103258144. PMID: 14994958.
  \end{itemize}
\item
  \textbf{Multidimensional Scale of Perceived Social Support (MSPSS)}

  \begin{itemize}
  \tightlist
  \item
    Gregory D. Zimet , Suzanne S. Powell , Gordon K. Farley , Sidney Werkman \& Karen A. Berkoff (1990) Psychometric Characteristics of the Multidimensional Scale of Perceived Social Support, Journal of Personality Assessment, 55:3-4, 610-617, DOI: 10.1080/00223891.1990.9674095
  \end{itemize}
\item
  \textbf{Mental Health Continuum Short Form (MHC-SF)}

  \begin{itemize}
  \tightlist
  \item
    Lamers S, Westerhof G, Bohlmeijer P, ten Klooster P, Keyes, C. Evaluating the psychometric properties of the mental health Continuum-Short Form (MHC-SF). Journal of Clinical Psychology. 2010. \url{https://doi.org/10.1002/jclp.20741}.
  \end{itemize}
\item
  \textbf{Toronoto Alexithymia Scale (external thinking subscale removed)}

  \begin{itemize}
  \tightlist
  \item
    R.Michael Bagby, James D.A. Parker, Graeme J. Taylor. The twenty-item Toronto Alexithymia scale---I. Item selection and cross-validation of the factor structure. Journal of Psychosomatic Research. 1994:38(1).https://doi.org/10.1016/0022-3999(94)90005-1.
  \end{itemize}
\item
  \textbf{Childhood Trauma Questionnaire Short Form (CTQ-SF)}

  \begin{itemize}
  \tightlist
  \item
    Hagborg JM, Kalin T, Gerdner A. The Childhood Trauma Questionnaire-Short Form (CTQ-SF) used with adolescents - methodological report from clinical and community samples. J Child Adolesc Trauma. 2022 Mar 30;15(4):1199-1213. doi: 10.1007/s40653-022-00443-8. PMID: 36439669; PMCID: PMC9684390.
  \end{itemize}
\item
  \textbf{Questionnaire of Unpredictabiltiy in Childhood - Brief Version (QUIC-5)}

  \begin{itemize}
  \tightlist
  \item
    Lindert N, Maxwell M, Liu S, Stern H, Baram T, Davis E, Risbrough V, Baker D, Nievergell C, Glynn L. Exposure to unpredictability and mental health: Validation of the brief version of the Questionnaire of Unpredictability in Childhood (QUIC-5) in English and Spanish. Frontiers in Psychology. 2022:13. \url{https://doi.org/10.3389/fpsyg.2022.971350}.
  \end{itemize}
\item
  \textbf{Assessment of Parent and Child Adversity (APCA)}

  \begin{itemize}
  \tightlist
  \item
    Edited by BAB Lab for adult respondents to capture childhood, adolescence, adulthood, and current occurances of stressful life events.
  \item
    King LS, Humphreys KL, Shaw GM, Stevenson DK, Gotlib IH. Validation of the Assessment of Parent and Child Adversity (APCA) in Mothers and Young Children. J Clin Child Adolesc Psychol. 2023 Sep 3;52(5):686-701. doi: 10.1080/15374416.2022.2042696. Epub 2022 May 2. PMID: 35500216; PMCID: PMC9626394.
  \end{itemize}
\item
  \textbf{Perceived Stress Scale (PSS)}

  \begin{itemize}
  \tightlist
  \item
    Cohen, S., Kamarck, T., \& Mermelstein, R. (1983). A Global Measure of Perceived Stress. Journal of Health and Social Behavior, 24(4), 385--396. \url{https://doi.org/10.2307/2136404}
  \end{itemize}
\end{itemize}

\hypertarget{study-contributions}{%
\section{Study Contributions}\label{study-contributions}}

FQ and BC designed study. FQ and GF collected data. FQ and GF (so far) carried out analysis.

\hypertarget{funding}{%
\section{Funding}\label{funding}}

ECHOES was funded through BC's startup fund from UCLA.

\hypertarget{data-collection}{%
\chapter{Data Collection}\label{data-collection}}

\hypertarget{recruitment}{%
\section{Recruitment}\label{recruitment}}

Recruitment was completed through Prolific.

\hypertarget{redcap}{%
\section{REDcap}\label{redcap}}

All data collection was completed on REDcap -- participants were redirected to the study's REDcap site.

\hypertarget{data-storage}{%
\section{Data Storage}\label{data-storage}}

Raw questionnaire data is stored on REDcap and on BAB Lab Box.

\hypertarget{data-analysis}{%
\chapter{Data Analysis}\label{data-analysis}}

\hypertarget{analysis-plans}{%
\section{Analysis Plans}\label{analysis-plans}}

\textbf{Dimensions of Adversity \& Alexithymia (Aim 1.1, 1.2)}: FQ and GF

We used CFA to test the a-priori specified 3-dimensional model of adversity, after assigning each adversity questionnaire item to a dimension according to theory, and removing duplicate items. Since fit to the data was not adequate (LIST CUTOFFS HERE), we performed EFA using promax rotation. We selected 3 factors for the EFA based on the Silouette and Parallel Plots methods. We then performed CFA on the model selected by EFA.

After obtaining factor scores for each participant from the CFA, we then used an OLS regression model to test associations of each factor (controlling of other factors, age, and sex) with alexithymia.

\hypertarget{analysis-scripts}{%
\section{Analysis Scripts}\label{analysis-scripts}}

Analysis scripts can be found at: {[}insert github repo link once public{]}

  \bibliography{book.bib,packages.bib,ref.bib}

\end{document}
